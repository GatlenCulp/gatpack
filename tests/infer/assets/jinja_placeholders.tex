% This file defines LaTeX commands for Jinja template visualization in editors
% These commands provide visual highlighting for Jinja template syntax while maintaining
% LaTeX compatibility. The commands use colored backgrounds to distinguish different
% template elements:
%   - \VAR: Light lavender background for variables (e.g., \VAR{user_name})
%   - \BLOCK: Misty rose background for control structures (e.g., \BLOCK{for item in items})
%   - \COMMENT: Light cyan background for comments
%
% These commands work in conjunction with the modified Jinja syntax for LaTeX templates
% to provide better editor support and syntax highlighting while preventing conflicts
% with LaTeX's curly brace syntax.
%
% Example usage:
%   \VAR{student_name}           % For variables
%   \BLOCK{for i in range(10)}   % For control structures
%   \COMMENT{TODO: Add more examples}  % For comments

\usepackage{xcolor}
\definecolor{jinjaColor}{HTML}{7B68EE}  % Medium slate blue color for Jinja
\definecolor{jinjaVarBg}{HTML}{E6E6FA}    % Light lavender for variables
\definecolor{jinjaBlockBg}{HTML}{FFE4E1}  % Misty rose for blocks
\definecolor{jinjaCommentBg}{HTML}{E0FFFF}  % Light cyan for comments
\newcommand{\VAR}[1]{\colorbox{jinjaVarBg}{\detokenize{#1}}}
\newcommand{\BLOCK}[1]{\colorbox{jinjaBlockBg}{\detokenize{#1}}}
\newcommand{\COMMENT}[1]{\colorbox{jinjaCommentBg}{\detokenize{#1}}} 